% La compilacion se debe hacer con la instruccion de PdfLatex
% para evitar problemas de tamaño de hoja.
% Todos los archivos de figuras deben estar en formato PDF.
% Al momento de imprimir el documento, se debe hacer sin escala de pagina.
\documentclass[letterpaper,12pt]{cicese}
%%%%%%%%%%%%%%%%%%%%%%%%%%%%%%%%%%%%%%%%%%%%%%%%%%%%%%%%%%%%%%%%%%%%%%%%%%%%%%%%%%%%%%%%%%%%%%%%%%%%%%%%%%%%%%%%%%%%%%%%%%%%%%%%%%%%%%%%%%%%%%%%%%%%%%%%%%%%%%%%%%%%%%%%%%%%%%%%%%%%%%%%%%%%%%%%%%%%%%%%%%%%%%%%%%%%%%%%%%%%%%%%%%%%%%%%%%%%%%%%%%%%%%%%%%%%
\usepackage{ marvosym }
\usepackage[letterpaper,left=3cm,right=2cm,top=2cm,bottom=2cm]{geometry}
%%\usepackage[ansinew]{inputenc}   %este paquete es para poner acentos directamente en el codigo de latex. sin barra
%%\usepackage[T1]{fontenc}
%\usepackage[spanish,english]{babel}
%%\usepackage[spanish,mexico,es-lcroman]{babel}
%paquete para insertar hipervinculos en referencias a capitulos, figuras, etc.
\usepackage{helvet}
\renewcommand{\familydefault}{\sfdefault} 
\usepackage[spanish, mexico,es-lcroman]{babel}
\usepackage[utf8x]{inputenc}
\usepackage{mdframed}
\usepackage[acronym]{glossaries}
\makeglossaries
\usepackage[linkcolor=blue]{hyperref}
\usepackage{cicese}
\usepackage{tabularx}
\usepackage{pbox}
\usepackage{makeidx}
\usepackage{eurosym}
\usepackage{amssymb}
\usepackage{amsmath}
\usepackage{gensymb}
\usepackage{graphicx} % figuras
\graphicspath{{figures//}}
\usepackage{emptypage} 
\usepackage{textcomp}
%\usepackage[linesnumbered,ruled,vlined]{algorithm2e}

\setlength{\parskip}{12pt plus 1pt minus 1pt}


% Paquete para pseudocodigo o algoritmos
%\usepackage[lined,boxed,boxruled,linesnumbered,spanish]{algorithm2e}
% Paquete para las graficas
\usepackage{float}
% Paquete para las enumeraciones
\usepackage{enumerate}
% Paquete para pseudocodigo o algoritmos
%este es el que venía:
%\usepackage[lined,boxed,boxruled,linesnumbered,spanish]{algorithm2e}

\usepackage{algorithm}
\usepackage{algorithmic}
\input{spanishAlgorithmic} % mi archivo de traducción

\usepackage{titlesec}
%\titleformat{\chapter}[display]
%  {\normalfont\huge\bfseries}{\chaptertitlename\ \thechapter}{16pt}{\Huge}
  
\titleformat{\chapter} % command
[hang] % shape
{\normalfont\bfseries\Large} % format
{\chaptertitlename\ \thechapter .} % label
{
16pt
} % sep
{
%    \rule{\textwidth}{1pt}
%    \vspace{1ex}
%    \centering
} % before-code
[
\vspace{-2ex}%
%\rule{\textwidth}{0.3pt}
] % after-code

%\titlespacing{command}{left spacing}{before spacing}{after spacing}[right]
\titlespacing\chapter{0pt}{-40.25pt plus 4pt minus 2pt}{12pt plus 2pt minus 2pt}
\titlespacing\section{0pt}{0pt plus 4pt minus 2pt}{3pt plus 2pt minus 2pt}
\titlespacing\subsection{0pt}{0pt plus 4pt minus 2pt}{3pt plus 2pt minus 2pt}
\titlespacing\subsubsection{0pt}{0pt plus 4pt minus 2pt}{3pt plus 2pt minus 2pt}

  	
% Paquete para estilizar los captions de tablas e imagenes
\usepackage[font=footnotesize,format=plain,labelfont=bf,up,textfont=bf,md,up,justification=justified]{caption}
\captionsetup{labelfont=bf}

% Para cambiar el color de fondo de una celda en una tabla
\usepackage[table]{xcolor}	
% Para rotar una tabla
\usepackage{rotating}
%\usepackage{subfig}
\usepackage{subfigure} % subfiguras
% Asignar espacio personalizado
\usepackage{setspace}
% Para hacer tablas de multiples páginas
\usepackage{longtable}
% Para utilizar multiples renglones en tablas
\usepackage{multirow}
\usepackage{booktabs}


%%%%%%%%%%%%%%%%%%%%%%%%%%%%%%%%%%%%
%para agregar teoremas, etc

\newtheorem{theorem}{Teorema}[chapter]
\newenvironment{proof}{\noindent{\bf Demostraci\'on.}}{\hfill$\blacksquare$} 
\newtheorem{lemma}{Lema}[chapter]
\newtheorem{teo}{Teorema}[chapter]
\newtheorem{obs}{Observaci\'on}[chapter]
\setcounter{MaxMatrixCols}{10}

\input{tcilatex}
\setcounter{secnumdepth}{3}
\setcounter{tocdepth}{3}
\begin{document}

% Escribe tu nombre, tal y como aparece en los registros
% de Posgrado. De ser necesario, separa con \- las sílabas
% de tu nombre
\autor{Dari\'en Alberto Miranda Boj\'orquez}

% Escribe el título de tu tesis (Si el titulo es mayor de 2 renglones es necesario
% cambiar el valor de espacio vertical para que conserve las dimensiones del margen.
% La modificacion se hace en el archivo cicese.sty en la seccion de la portada que
% ya esta indicado las dimensiones que debe tener para 3 renglones.)
\titulo{Detecci\'on de ansiedad en cuidadores de personas con demencia por medio de c\'omputo vestible}

% Para el resumen, escribe aqui el título en inglés
\tituloIN{Anxiety detection on caregivers of people with dementia trough wearable computing}

% Aquí va el nombre de tu programa de posgrado
\posgrado{Ciencias de la Computación}
\posgradoUP{CIENCIAS DE LA COMPUTACIÓN}
%con orientaci\'on en \uppercase\expandafter{C\'omputo Biomolecular}
% Aquí va el nombre de orientacion para el que aplique (sino comentarlo)
%\orientacion{}

% Escribe aquí el nombre del posgrado en inglés
\posgradoIN{Computer Science}
%with orientation in \uppercase\expandafter{Biomolecular Computing}
% Escribe aquí el nombre de orientacion para el que aplique (sino comentarlo)
\orientacionIN{}

% El nombre del grado en inglés
\gradoIN{Master in Computer Science}

% Este comando especifica si la tesis es de maestria (usa \mc) o de
% doctorado (usa \dc)
\mc

% Fecha (mes y año) de la defensa de tesis, en español y en ingles
\fechaexamen{Septiembre, 2015}\fechaexamenIN{September, 2015}

% Fecha (dia, mes y año) de la defensa de tesis para la hoja de firmas
\fechaexamencompleta{Septiembre, 2015}

% Año de la defensa //Este valor a aparecerá en la parte inferior de la primer página de la tesis
\anioDefensa{2015}

% Nombre de tu director de tesis, si tienes codirector repite el comando
\director{Dr. Jes\'us Favela Vara}
%\director{Dr. Asesor 2}

% Nombres de los sinodales, en el orden que aparecerán en la
% página de firmas
%%NO UTILIZAR ABREVIATURAS
\sinodal{Dra. Tentori Espinosa M\'onica Elizabeth}
\sinodal{Dr. Chavez Gonz\'alez Edgar Leonel}
\sinodal{Dra. Herzka Llona Sharon Zinah}
\sinodal{Dra. Cordero Esquivel Esquivel}
%\sinodal{Dr. 4}
%\sinodal{Dr. 5}
%\sinodal{Dr. 6}

% Nombre del Director de Estudios de Posgrado
\dirposgrado{Dr. Jesús Favela Vara}
% Nombre del Coordinador del Programa de Posgrado
\cpp{Dra. Ana Isabel Martínez García}

% Escribe aqui el nombre del archivo .tex que contiene el texto del
% resumen en español. Omite escribir la extensión .tex
\resumenES{Resumen}

% Palabras clave del resumen en español
\palabrasclave{Detecci\'on de ansiedad, Cuidadores de personas con demencia, C\'omputo vestible, Se\~nales fisiol\'ogicas}

% Escribe aqui el nombre del archivo .tex que contiene el texto del
% resumen en inglés. Omite escribir la extensión .tex
\resumenIN{Abstract}

% Palabras clave del resumen en inglés
\keywords{Anxiety detection, Caregivers of people with dementia, Wearable computing, Physiological signals}

% Nombre del archivo .tex con el texto de la dedicatoria, omite
% la extensión .tex
\dedicatoria{Dedicatoria}

% Nombre del archivo .tex con el texto de los agradecimientos, omite
% la extensión .tex
\agradecimientos{Agradecimientos}

% Este comando crea todas las páginas preliminares de la tesis
\preliminares

{\normalsize
%TCIMACRO{\QSubDoc{Include Capitulo01}{
\chapter{Introducci\'on}\label{capit:cap1}
\vspace{-2.0325ex}%
\noindent
\rule{\textwidth}{0.5pt}
\vspace{-5.5ex}% 
\newcommand{\pushline}{\Indp}


Actualmente la poblaci\'on de adutos mayores va en aumento. Cuando sufren de enfermedades o padecimiento es com\'un que necesiten de servicios de cuidadores. Los cuidadores por su parte, sufren de problemas derivados de su labor. En un estudio reciente, 60\% de los cuidadores desarrollaron un desorden depresivo y/o de ansiedad en los primeros 24 meses: 37\% de depresi\'on, 55\% desorden de ansiedad y 32\% ambos \citep{Joling2014}. Este mismo problema es mas pronunciado en los cuidadores de personas con demencia.  Casi un cuarto de ellos tienen un nivel de ansiedad cl\'inico significante \citep{Cooper200615}.

	En este trabajo, se explora la utilidad del c\'omputo vestible (computadoras lo suficientemente peque\~nas para ser usadas como ropa) como herramienta para la detecci\'on de ansiedad en cuidadores de personas con demencia. Se presenta el dise\~no de un experimento para recabar datos que ayuden a la detecci\'on de ansiedad y los resultados preliminares de la estimaci\'on utilizando aprendizaje de m\'aquina.
	
	
	
	%	Casi un cuarto de ellos tienen un nivel de ansiedad cl\'inico significante \citep{Cooper200615}. Mientras que la carga de los cuidadores informales aumenta, se vuelve mas probable que sufran de ansiedad y depresi\'on \citep{Denno20131731}. La carga en los cuidadores (f\'isica o psicol\'ogica) podr\'ia aumentar los niveles de ansiedad. Entre mas demandante es un servicio, mayor podr\'ia ser la ansiedad percibida. 
	
	

	
	%Los comportamientos bizarros o impredecibles de la persona afectada por demencia aumentan la carga emocional \citep{Rosa201054}. En este estudio, se har\'a uso del c\'omputo vestible para capturar informaci\'on de ansiedad y se utilizar\'a aprendizaje de m\'aquina para detectar periodos de ansiedad en cuidadores de personas con demencia.

	%La demencia es un s\'indrome del declive de las habilidades cognitivas. Los s\'intomas comunes son: problemas de memoria, dificultades para realizar tareas, mal juicio, deterioro del lenguaje hablado y cambios de humor\citep{Aziz}. Afecta alrededor del 4\% de las personas mayores de 65 a\~nos y al 40\% de las personas mayores de 90. 
\section{Plantamiento del problema}
Los cuidadores sufren de problemas de ansiedad, estr\'es y depresi\'on[ref]. A pesar de que se ha demostrado en otros estudios la capacidad de utilizar sensores de datos fisiol\'ogicos para detectar ansiedad, no se ha abordado el problema de los cuidadores de personas con demencia. Los retos para recabar datos de estas situaciones son variados. Tener certeza que la persona se encuentra en un estado ansioso y la captura de datos fisiol\'ogicos en situaciones naturales son los retos m\'as significativos. En este trabajo, se propone un experimento para solventar estos problemas y se presenta un an\'alisis preliminar de dichos datos
\section{Objectivos}
	A continuaci\'on se listan los objetivos de este trabajo.
\subsection{Objetivo general}
	Detectar periodos de ansiedad en cuidadores de personas con demencia por medio de c\'omputo vestible
\subsection{Objetivos espec\'ificos}
	\begin{enumerate}
		\item Dise\~nar un experimento para la detecci\'on de ansiedad en cuidadores de personas con demencia.
		\item Realizar el experimento para recabar datos fisiol\'ogicos que puedan ser usados para la detecci\'on de ansiedad.
		\item Analizar la informaci\'on fisiol\'ogica recabada para conocer si es factible estimar ansiedad a partir de estos datos.
	\end{enumerate}
\subsection{Pregunta de investigaci\'on}
	?`Como pueden los dispositivos vestibles ayudar a la detecci\'on de ansiedad en cuidadores de personas con demencia?
\subsection{Propuesta de la soluci\'on}
	Se propone dise\~nar un experimento para recabar datos de se\~nales fisiolo\'ogicas en cuidadores de persona con demencia en situaciones naturalistas. Se pretender recabar los datos en situaciones naturalistas, es decir, situaciones cercanas a las que los cuidadores enfrentan. Adem\'as se utilizar\'an datos en forma de video, autoreportado y observaci\'on para determinar en que situaciones el participante siente los periodos de ansiedad. Por \'ultimo, se utilizar\'an t\'ecnicas de aprendizaje de m\'aquina para evaluar la factibilidad de utilizar estos datos para la detecci\'on de ansiedad.

\section{Metodolog\'ia }\label{secc:methodology}
La metodolog\'ia que las aplicaciones conscientes de la ansiedad deber\'ian de sguir consiste de tres pasos principales: \textit{Captura de datos}, \textit{Detecci\'on de ansiedad}, y un paso hipot\'etico de \textit{Estrategias de afrontamiento} (ver figura ~\ref{fig:metodology}). El primer paso consiste en recabar informaci\'on de cuidadores bajo situaciones de ansiedad con el fin de etiquetar eventos. Esta captura se logr\'o por medio de un experimento que implementa una t\'ecnica de ``Naturalistic Enactment (NE)'' \citep{Castro11}.
%Los eventos ser\'an clasificados como de ansiedad y no ansiedad. Se segmentar\'an las se\~nales en periodos de 30 segundos y se extraer\'an las caracter\'isticas de las porciones de las se\~nales correspondientes. Los datos ser\'an procesados con utiler\'ias ya programadas en python. Una vez capturada y etiquetada la informaci\'on, se desarroll\'o un m\'etodo basado en t\'ecnicas de aprendizaje de m\'aquina que tom\'o los segmentos de ansiedad.

El tercer paso, se discute como las posibles aplicaciones por medio de tecnolog\'ia m\'ovil y/o pantallas ambientales.
\begin{figure}[h!]
        \centering
        \subfigure[]{\includegraphics[width=160mm]{./Figures/img_metodologia}}
        \caption{Metodolog\'ia seguida durante la investigaci\'on.} \label{fig:metodology}
\end{figure}
\subsection{Contribuci\'on}
	En la actualidad, no existen intervenciones no tecnol\'ogicas para detectar la ansiedad en cuidadores de personas con demencia. Estas intervenciones tradicionales se basan en el uso de encuestas y cuestionarios y actuan sobre periodos de tiempo muy largos. Este trabajo se basa en la detecci\'on de ansiedad situacional, que permite conocer con m\'as a detalle la frecuencia de estos eventos.
	Adem\'as, se generar\'a una base de datos de eventos de ansiedad que permitir\'a realizar pruebas de t\'ecnicas de detecci\'on de ansiedad que surgan en el futuro.
%\section{Escenario}
%	Los participantes ser\'an transportados del departamento de computaci\'on de CICESE hacia una casa donde se encontrar\'a el adulto mayor
%	para realizar la prueba.
%	Se les pedir\'a reposar durante 15 minutos en el lugar para regularizar su sudoraci\'on y ritmo cardiaco.
%	Luego, se les tomar\'a una muestra de 5 minutos de sus se\~nales fisiol\'ogicas como l\'inea base por medio de una banda para HR, una pulsera para GSR y una diadema para EEG. Durante estos 5 minutos
%	se les pedir\'a  que reposen sentados en una habitaci\'on sin el adulto mayor y con los ojos cerrados, concetr\'andose en su respiraci\'on como t\'ecnica de relajaci\'on.

%	Pasado el tiempo de relajaci\'on se les presentar\'a al adulto mayor en una habitaci\'on diferente.Despu\'es de la introducci\'on llevar\'an a cabo la terapia. Una vez mas, llevar\'an sobre el cuerpo la banda de ritmo cardiaco, la pulsera para GSR y la diadema de EEG. Durante la prueba, se les pedir\'a que reporten su nivel de ansiedad por medio de un formato especial.

%\section{Configuraci\'on del escenario}
	
%	Los datos ser\'an capturados de la siguiente manera:
%	\begin{itemize}
%		\item{La se\~nal de suduraci\'on se obtendr\'a por medio de la pulsera \textbf{Empatica E3} que ser\'a conectada a un tel\'efono inteligente Samsung S4 con android 4.0 ejecutando la aplicaci\'on CareMeToo hecha en el laboratorio.}
%		\item{La se\~nal de ritmo cardiaco se obtendr\'a por medio de una banda zephyr HxM conectado a una macbook 2008 ejecutando la aplicaci\'on \textbf{anxiLogger} hecha en el laboratorio.}
%		\item{La se\~nal de EEG se obtendr\'a por medio de la diadema Muse conecatada a la misma macbook 2008 ejecutando la aplicaci\'on Muse Lab.}
%		\item{Todo el procedimiento ser\'a grabado por medio de una c\'amara Sony HD instalada en el lugar.}
%	\end{itemize}
%\section{Consideraciones}
%	\begin{itemize}
%		\item{Los participantes no podr\'an tomar bebidas con cafe\'ina (caf\'e, t\'e, refresco, bebidas energ\'eticas etc.) durante al menos 8 horas antes de la prueba.}
%		\item{Los participantes no podr\'an interactuar con los investigadores una vez que inicie la prueba.}
%		\item{El papel del adulto mayor ser\'a realizado por una actriz profesional con experiencia en papeles de adultos mayores y aconsejada en el comportamiento de una persona con demencia por una profesional.}
%		\item{Los participantes ser\'an previamente entrenados en la aplicaci\'on de la terapia unos dias antes de la prueba.}
%	\end{itemize}


\subsection{Organizaci\'on de la tesis}
	El documento se encuentra dividido en cinco cap\'itulos y dos ap\'endices. A continuaci\'on se describe el contenido de cada uno de ellos.
	
	En el \textbf{CAP\'ITULO 2} \textit{(Fundamentos te\'oricos)} se abordan los temas correspondientes a la definici\'on de la ansiedad y sus efectos en el cuidador. Tambi\'en se explican las t\'ecnicas tradicionales y tecnol\'ogicas para medir la ansiedad en general.

	En el \textbf{CAP\'ITULO 3} \textit{(Dise\~no de un experimento para inducir ansiedad en cuidadores de personas con demencia)} se explica el dise\~no del experimento que permiti\'o recabar datos de ansiedad en cuidadores.

	En el \textbf{CAP\'ITULO 4} \textit{(Resultados del experimento y detecci\'on de ansiedad)} se muestran los resultados del experimento y los m\'etodos utilizados para la detecci\'on de ansiedad, as\'i como los resultados preliminares de clasificaci\'on.

	En el \textbf{CAP\'ITULO 5} \textit{(Conclusiones y trabajo a futuro)} se presentan las conclusiones, limitaciones y direcciones para trabajos futuros.

	En el \textbf{Ap\'endice A} \textit{(Instrumentos y protocolos para la detecci\'on de ansiedad)} se incluyen los formatos de consentimiento, autoreportado y carta de no divulgaci\'on utilizados durante el experimento.

	En el \textbf{Ap\'endice B} \textit{(Detalles de resultados: Tablas y Figuras)} se incluyen a detalle los resultados del experimento y pruebas realizados.


\subsection{Conclusi\'on}
	Se encuentra documentada a la ansiedad como un problema que enfrentan los cuidadores de personas con demencia. Con la tendencia en el aumento de la poblaci\'on de adultos mayores, se espera que el n\'umero de cuidadores aumente por lo que es necesario tomar en cuenta las necesidades de este sector. Con la tendencia en el uso de dispositivos vestibles capaces de sensar el ambiente del usuario, la detecci\'on de ansiedad por medio de c\'omputo vestible se vuelve una opci\'on para mejorar la calidad de vida de los cuidadores.

\newpage
%%=====================================================
} }%
%BeginExpansion

\chapter{Introducci\'on}\label{capit:cap1}
\vspace{-2.0325ex}%
\noindent
\rule{\textwidth}{0.5pt}
\vspace{-5.5ex}% 
\newcommand{\pushline}{\Indp}


Actualmente la poblaci\'on de adutos mayores va en aumento. Cuando sufren de enfermedades o padecimiento es com\'un que necesiten de servicios de cuidadores. Los cuidadores por su parte, sufren de problemas derivados de su labor. En un estudio reciente, 60\% de los cuidadores desarrollaron un desorden depresivo y/o de ansiedad en los primeros 24 meses: 37\% de depresi\'on, 55\% desorden de ansiedad y 32\% ambos \citep{Joling2014}. Este mismo problema es mas pronunciado en los cuidadores de personas con demencia.  Casi un cuarto de ellos tienen un nivel de ansiedad cl\'inico significante \citep{Cooper200615}.

	En este trabajo, se explora la utilidad del c\'omputo vestible (computadoras lo suficientemente peque\~nas para ser usadas como ropa) como herramienta para la detecci\'on de ansiedad en cuidadores de personas con demencia. Se presenta el dise\~no de un experimento para recabar datos que ayuden a la detecci\'on de ansiedad y los resultados preliminares de la estimaci\'on utilizando aprendizaje de m\'aquina.
	
	
	
	%	Casi un cuarto de ellos tienen un nivel de ansiedad cl\'inico significante \citep{Cooper200615}. Mientras que la carga de los cuidadores informales aumenta, se vuelve mas probable que sufran de ansiedad y depresi\'on \citep{Denno20131731}. La carga en los cuidadores (f\'isica o psicol\'ogica) podr\'ia aumentar los niveles de ansiedad. Entre mas demandante es un servicio, mayor podr\'ia ser la ansiedad percibida. 
	
	

	
	%Los comportamientos bizarros o impredecibles de la persona afectada por demencia aumentan la carga emocional \citep{Rosa201054}. En este estudio, se har\'a uso del c\'omputo vestible para capturar informaci\'on de ansiedad y se utilizar\'a aprendizaje de m\'aquina para detectar periodos de ansiedad en cuidadores de personas con demencia.

	%La demencia es un s\'indrome del declive de las habilidades cognitivas. Los s\'intomas comunes son: problemas de memoria, dificultades para realizar tareas, mal juicio, deterioro del lenguaje hablado y cambios de humor\citep{Aziz}. Afecta alrededor del 4\% de las personas mayores de 65 a\~nos y al 40\% de las personas mayores de 90. 
\section{Plantamiento del problema}
Los cuidadores sufren de problemas de ansiedad, estr\'es y depresi\'on[ref]. A pesar de que se ha demostrado en otros estudios la capacidad de utilizar sensores de datos fisiol\'ogicos para detectar ansiedad, no se ha abordado el problema de los cuidadores de personas con demencia. Los retos para recabar datos de estas situaciones son variados. Tener certeza que la persona se encuentra en un estado ansioso y la captura de datos fisiol\'ogicos en situaciones naturales son los retos m\'as significativos. En este trabajo, se propone un experimento para solventar estos problemas y se presenta un an\'alisis preliminar de dichos datos
\section{Objectivos}
	A continuaci\'on se listan los objetivos de este trabajo.
\subsection{Objetivo general}
	Detectar periodos de ansiedad en cuidadores de personas con demencia por medio de c\'omputo vestible
\subsection{Objetivos espec\'ificos}
	\begin{enumerate}
		\item Dise\~nar un experimento para la detecci\'on de ansiedad en cuidadores de personas con demencia.
		\item Realizar el experimento para recabar datos fisiol\'ogicos que puedan ser usados para la detecci\'on de ansiedad.
		\item Analizar la informaci\'on fisiol\'ogica recabada para conocer si es factible estimar ansiedad a partir de estos datos.
	\end{enumerate}
\subsection{Pregunta de investigaci\'on}
	?`Como pueden los dispositivos vestibles ayudar a la detecci\'on de ansiedad en cuidadores de personas con demencia?
\subsection{Propuesta de la soluci\'on}
	Se propone dise\~nar un experimento para recabar datos de se\~nales fisiolo\'ogicas en cuidadores de persona con demencia en situaciones naturalistas. Se pretender recabar los datos en situaciones naturalistas, es decir, situaciones cercanas a las que los cuidadores enfrentan. Adem\'as se utilizar\'an datos en forma de video, autoreportado y observaci\'on para determinar en que situaciones el participante siente los periodos de ansiedad. Por \'ultimo, se utilizar\'an t\'ecnicas de aprendizaje de m\'aquina para evaluar la factibilidad de utilizar estos datos para la detecci\'on de ansiedad.

\section{Metodolog\'ia }\label{secc:methodology}
La metodolog\'ia que las aplicaciones conscientes de la ansiedad deber\'ian de sguir consiste de tres pasos principales: \textit{Captura de datos}, \textit{Detecci\'on de ansiedad}, y un paso hipot\'etico de \textit{Estrategias de afrontamiento} (ver figura ~\ref{fig:metodology}). El primer paso consiste en recabar informaci\'on de cuidadores bajo situaciones de ansiedad con el fin de etiquetar eventos. Esta captura se logr\'o por medio de un experimento que implementa una t\'ecnica de ``Naturalistic Enactment (NE)'' \citep{Castro11}.
%Los eventos ser\'an clasificados como de ansiedad y no ansiedad. Se segmentar\'an las se\~nales en periodos de 30 segundos y se extraer\'an las caracter\'isticas de las porciones de las se\~nales correspondientes. Los datos ser\'an procesados con utiler\'ias ya programadas en python. Una vez capturada y etiquetada la informaci\'on, se desarroll\'o un m\'etodo basado en t\'ecnicas de aprendizaje de m\'aquina que tom\'o los segmentos de ansiedad.

El tercer paso, se discute como las posibles aplicaciones por medio de tecnolog\'ia m\'ovil y/o pantallas ambientales.
\begin{figure}[h!]
        \centering
        \subfigure[]{\includegraphics[width=160mm]{./Figures/img_metodologia}}
        \caption{Metodolog\'ia seguida durante la investigaci\'on.} \label{fig:metodology}
\end{figure}
\subsection{Contribuci\'on}
	En la actualidad, no existen intervenciones no tecnol\'ogicas para detectar la ansiedad en cuidadores de personas con demencia. Estas intervenciones tradicionales se basan en el uso de encuestas y cuestionarios y actuan sobre periodos de tiempo muy largos. Este trabajo se basa en la detecci\'on de ansiedad situacional, que permite conocer con m\'as a detalle la frecuencia de estos eventos.
	Adem\'as, se generar\'a una base de datos de eventos de ansiedad que permitir\'a realizar pruebas de t\'ecnicas de detecci\'on de ansiedad que surgan en el futuro.
%\section{Escenario}
%	Los participantes ser\'an transportados del departamento de computaci\'on de CICESE hacia una casa donde se encontrar\'a el adulto mayor
%	para realizar la prueba.
%	Se les pedir\'a reposar durante 15 minutos en el lugar para regularizar su sudoraci\'on y ritmo cardiaco.
%	Luego, se les tomar\'a una muestra de 5 minutos de sus se\~nales fisiol\'ogicas como l\'inea base por medio de una banda para HR, una pulsera para GSR y una diadema para EEG. Durante estos 5 minutos
%	se les pedir\'a  que reposen sentados en una habitaci\'on sin el adulto mayor y con los ojos cerrados, concetr\'andose en su respiraci\'on como t\'ecnica de relajaci\'on.

%	Pasado el tiempo de relajaci\'on se les presentar\'a al adulto mayor en una habitaci\'on diferente.Despu\'es de la introducci\'on llevar\'an a cabo la terapia. Una vez mas, llevar\'an sobre el cuerpo la banda de ritmo cardiaco, la pulsera para GSR y la diadema de EEG. Durante la prueba, se les pedir\'a que reporten su nivel de ansiedad por medio de un formato especial.

%\section{Configuraci\'on del escenario}
	
%	Los datos ser\'an capturados de la siguiente manera:
%	\begin{itemize}
%		\item{La se\~nal de suduraci\'on se obtendr\'a por medio de la pulsera \textbf{Empatica E3} que ser\'a conectada a un tel\'efono inteligente Samsung S4 con android 4.0 ejecutando la aplicaci\'on CareMeToo hecha en el laboratorio.}
%		\item{La se\~nal de ritmo cardiaco se obtendr\'a por medio de una banda zephyr HxM conectado a una macbook 2008 ejecutando la aplicaci\'on \textbf{anxiLogger} hecha en el laboratorio.}
%		\item{La se\~nal de EEG se obtendr\'a por medio de la diadema Muse conecatada a la misma macbook 2008 ejecutando la aplicaci\'on Muse Lab.}
%		\item{Todo el procedimiento ser\'a grabado por medio de una c\'amara Sony HD instalada en el lugar.}
%	\end{itemize}
%\section{Consideraciones}
%	\begin{itemize}
%		\item{Los participantes no podr\'an tomar bebidas con cafe\'ina (caf\'e, t\'e, refresco, bebidas energ\'eticas etc.) durante al menos 8 horas antes de la prueba.}
%		\item{Los participantes no podr\'an interactuar con los investigadores una vez que inicie la prueba.}
%		\item{El papel del adulto mayor ser\'a realizado por una actriz profesional con experiencia en papeles de adultos mayores y aconsejada en el comportamiento de una persona con demencia por una profesional.}
%		\item{Los participantes ser\'an previamente entrenados en la aplicaci\'on de la terapia unos dias antes de la prueba.}
%	\end{itemize}


\subsection{Organizaci\'on de la tesis}
	El documento se encuentra dividido en cinco cap\'itulos y dos ap\'endices. A continuaci\'on se describe el contenido de cada uno de ellos.
	
	En el \textbf{CAP\'ITULO 2} \textit{(Fundamentos te\'oricos)} se abordan los temas correspondientes a la definici\'on de la ansiedad y sus efectos en el cuidador. Tambi\'en se explican las t\'ecnicas tradicionales y tecnol\'ogicas para medir la ansiedad en general.

	En el \textbf{CAP\'ITULO 3} \textit{(Dise\~no de un experimento para inducir ansiedad en cuidadores de personas con demencia)} se explica el dise\~no del experimento que permiti\'o recabar datos de ansiedad en cuidadores.

	En el \textbf{CAP\'ITULO 4} \textit{(Resultados del experimento y detecci\'on de ansiedad)} se muestran los resultados del experimento y los m\'etodos utilizados para la detecci\'on de ansiedad, as\'i como los resultados preliminares de clasificaci\'on.

	En el \textbf{CAP\'ITULO 5} \textit{(Conclusiones y trabajo a futuro)} se presentan las conclusiones, limitaciones y direcciones para trabajos futuros.

	En el \textbf{Ap\'endice A} \textit{(Instrumentos y protocolos para la detecci\'on de ansiedad)} se incluyen los formatos de consentimiento, autoreportado y carta de no divulgaci\'on utilizados durante el experimento.

	En el \textbf{Ap\'endice B} \textit{(Detalles de resultados: Tablas y Figuras)} se incluyen a detalle los resultados del experimento y pruebas realizados.


\subsection{Conclusi\'on}
	Se encuentra documentada a la ansiedad como un problema que enfrentan los cuidadores de personas con demencia. Con la tendencia en el aumento de la poblaci\'on de adultos mayores, se espera que el n\'umero de cuidadores aumente por lo que es necesario tomar en cuenta las necesidades de este sector. Con la tendencia en el uso de dispositivos vestibles capaces de sensar el ambiente del usuario, la detecci\'on de ansiedad por medio de c\'omputo vestible se vuelve una opci\'on para mejorar la calidad de vida de los cuidadores.

\newpage
%%=====================================================

%EndExpansion
\newpage }

{\normalsize
%TCIMACRO{\QSubDoc{Include Capitulo02}{\chapter{Much title}\label{capit:cap2}
\vspace{-2.0325ex}%
\noindent
\rule{\textwidth}{0.5pt}
\vspace{-5.5ex}% 
\newcommand{\pushline}{\Indp}% Indent puede ir o no :p

\section{More stuff goes here}\label{secc:introduccion}
Lorem ipsum dolor sit amet, consectetur adipiscing elit. Aliquam sit amet lobortis turpis. Praesent auctor mi metus, sed bibendum ligula efficitur eu. Suspendisse ut ante id erat interdum accumsan. Pellentesque eget hendrerit eros, et ullamcorper elit. Proin a lacus et sem hendrerit efficitur. Praesent eget eros sed tellus dapibus bibendum sit amet vel justo. Maecenas finibus porttitor dictum. Fusce lacinia dictum interdum. 
 
\subsubsection{Another title}\label{secc:fatiga}
El significado etimológico de fatiga proviene del latín \textit{fatigare}; \textit{fatim} que significa ''con exceso'', y \textit{agere} que significa ''hacer''. Es típicamente definida como la reducción en la capacidad fisiológica de un tejido u órgano con manifestación física y/o psíquica generada por la demanda prolongada de \textbf{actividad física y/o mental}; respectivamente.

Lorem ipsum dolor sit amet, consectetur adipiscing elit. Aliquam sit amet lobortis turpis. Praesent auctor mi metus, sed bibendum ligula efficitur eu. Suspendisse ut ante id erat interdum accumsan. Pellentesque eget hendrerit eros, et ullamcorper elit. Proin a lacus et sem hendrerit efficitur. Praesent eget eros sed tellus dapibus bibendum sit amet vel justo. Maecenas finibus porttitor dictum. Fusce lacinia dictum interdum \ref{secc:introduccion}

\begin{figure}%[htbp]
\centering
\subfigure[Mi metus, sed bibendum ligula efficitur eu.]{\includegraphics[width=80mm]{./Figures/instauracionFatigaReposo_1_3}} 
\subfigure[Mi metus, sed bibendum ligula efficitur eu.]{\includegraphics[width=80mm]{./Figures/instauracionFatigaReposo_1_3}}
\subfigure[Mi metus, sed bibendum ligula efficitur eu.]{\includegraphics[width=160mm]{./Figures/instauracionFatigaReposo_1_3}}
\caption{Lorem ipsum dolor sit amet, consectetur adipiscing elit. Aliquam sit amet lobortis turpis. Praesent auctor mi metus.  } \label{fig:instauracionFatigaReposo}
\end{figure}





\newpage
%%=====================================================
} }%
%BeginExpansion
\chapter{Much title}\label{capit:cap2}
\vspace{-2.0325ex}%
\noindent
\rule{\textwidth}{0.5pt}
\vspace{-5.5ex}% 
\newcommand{\pushline}{\Indp}% Indent puede ir o no :p

\section{More stuff goes here}\label{secc:introduccion}
Lorem ipsum dolor sit amet, consectetur adipiscing elit. Aliquam sit amet lobortis turpis. Praesent auctor mi metus, sed bibendum ligula efficitur eu. Suspendisse ut ante id erat interdum accumsan. Pellentesque eget hendrerit eros, et ullamcorper elit. Proin a lacus et sem hendrerit efficitur. Praesent eget eros sed tellus dapibus bibendum sit amet vel justo. Maecenas finibus porttitor dictum. Fusce lacinia dictum interdum. 
 
\subsubsection{Another title}\label{secc:fatiga}
El significado etimológico de fatiga proviene del latín \textit{fatigare}; \textit{fatim} que significa ''con exceso'', y \textit{agere} que significa ''hacer''. Es típicamente definida como la reducción en la capacidad fisiológica de un tejido u órgano con manifestación física y/o psíquica generada por la demanda prolongada de \textbf{actividad física y/o mental}; respectivamente.

Lorem ipsum dolor sit amet, consectetur adipiscing elit. Aliquam sit amet lobortis turpis. Praesent auctor mi metus, sed bibendum ligula efficitur eu. Suspendisse ut ante id erat interdum accumsan. Pellentesque eget hendrerit eros, et ullamcorper elit. Proin a lacus et sem hendrerit efficitur. Praesent eget eros sed tellus dapibus bibendum sit amet vel justo. Maecenas finibus porttitor dictum. Fusce lacinia dictum interdum \ref{secc:introduccion}

\begin{figure}%[htbp]
\centering
\subfigure[Mi metus, sed bibendum ligula efficitur eu.]{\includegraphics[width=80mm]{./Figures/instauracionFatigaReposo_1_3}} 
\subfigure[Mi metus, sed bibendum ligula efficitur eu.]{\includegraphics[width=80mm]{./Figures/instauracionFatigaReposo_1_3}}
\subfigure[Mi metus, sed bibendum ligula efficitur eu.]{\includegraphics[width=160mm]{./Figures/instauracionFatigaReposo_1_3}}
\caption{Lorem ipsum dolor sit amet, consectetur adipiscing elit. Aliquam sit amet lobortis turpis. Praesent auctor mi metus.  } \label{fig:instauracionFatigaReposo}
\end{figure}





\newpage
%%=====================================================

%EndExpansion
\newpage }

{\normalsize
%TCIMACRO{\QSubDoc{Include Capitulo03}{\chapter{Metodolog\'ia y dise\~no de un m\'etodo para inducir ansiedad en cuidadores de personas con demencia}\label{capit:cap3}
\vspace{-2.0325ex}%
\noindent
\rule{\textwidth}{0.5pt}
\vspace{-5.5ex}% 
\newcommand{\pushline}{\Indp}% Indent puede ir o no :p
\section{Introducci\'on}\label{secc:introduction}

Como vimos en el cap\'itulo 2, la mayor\'ia de los estudios logran inducir ansiedad o estr\'es por medio de situaciones controladas dentro del laboratorio exitosamente. Sin embargo, generar ansiedad en cuidadores informales es mucho mas dif\'icil: El escenario de un laboratorio no coincide con el entorno en el que una persona con demencia se desarrolla por lo que los comportamientos impredecibles no ser\'ian congruentes con el ambiente. Adem\'as, exponer a personas sin experiencia ante una persona con demencia que tiene necesidades reales resultar\'ia riesgoso para ambos individuos. Por otra parte, realizar una intervenci\'on totalmente natural a\~nade un grado de dificultad al estudio, resultando en ruido en los datos recolectados (p. ej. la se\~nal de ritmo card\'iaco podr\'ia ser alta no por una situaci\'on de ansiedad, sino por una actividad f\'isica, m\'ultiples distracciones o responsabilidades al mismo tiempo para el cuidador) haciendo d\'ificil de analizarlos para probar hip\'otesis.

A continuaci\'on, se muestra un experimento que implementa la  t\'ecnica conocida como ``Naturalistic enactment (NE)'' \cite{Castro11}


\section{Un experimento para inducir ansiedad en cuidadores informales}
\begin{figure}[h]
        \centering
        \subfigure[]{\includegraphics[width=160mm,height=80mm]{./Figures/img_exp_setup}}
        \caption{Configuraci\'on del escenario del experimento} \label{fig:img_exp_setup}
\end{figure}

\newpage
%%=====================================================

} }%
%BeginExpansion
\chapter{Metodolog\'ia y dise\~no de un m\'etodo para inducir ansiedad en cuidadores de personas con demencia}\label{capit:cap3}
\vspace{-2.0325ex}%
\noindent
\rule{\textwidth}{0.5pt}
\vspace{-5.5ex}% 
\newcommand{\pushline}{\Indp}% Indent puede ir o no :p
\section{Introducci\'on}\label{secc:introduction}

Como vimos en el cap\'itulo 2, la mayor\'ia de los estudios logran inducir ansiedad o estr\'es por medio de situaciones controladas dentro del laboratorio exitosamente. Sin embargo, generar ansiedad en cuidadores informales es mucho mas dif\'icil: El escenario de un laboratorio no coincide con el entorno en el que una persona con demencia se desarrolla por lo que los comportamientos impredecibles no ser\'ian congruentes con el ambiente. Adem\'as, exponer a personas sin experiencia ante una persona con demencia que tiene necesidades reales resultar\'ia riesgoso para ambos individuos. Por otra parte, realizar una intervenci\'on totalmente natural a\~nade un grado de dificultad al estudio, resultando en ruido en los datos recolectados (p. ej. la se\~nal de ritmo card\'iaco podr\'ia ser alta no por una situaci\'on de ansiedad, sino por una actividad f\'isica, m\'ultiples distracciones o responsabilidades al mismo tiempo para el cuidador) haciendo d\'ificil de analizarlos para probar hip\'otesis.

A continuaci\'on, se muestra un experimento que implementa la  t\'ecnica conocida como ``Naturalistic enactment (NE)'' \cite{Castro11}


\section{Un experimento para inducir ansiedad en cuidadores informales}
\begin{figure}[h]
        \centering
        \subfigure[]{\includegraphics[width=160mm,height=80mm]{./Figures/img_exp_setup}}
        \caption{Configuraci\'on del escenario del experimento} \label{fig:img_exp_setup}
\end{figure}

\newpage
%%=====================================================


%EndExpansion
\newpage }

{\normalsize
%TCIMACRO{\QSubDoc{Include Capitulo04}{
\chapter{Evaluacion}\label{capit:cap4}
\vspace{-2.0325ex}%
\noindent
\rule{\textwidth}{0.5pt}
\vspace{-5.5ex}% 
\newcommand{\pushline}{\Indp}% Indent puede ir o no :p

\section{Introducci\'on}\label{cap4:intro}

\section{Codificaci\'on}

\section{Extracci\'on de caracter\'isticas}
Por cada tipo de se\~nal, se tomaron varias caracter\'isticas que generalizan un segmento de datos. La tabla ~\ref{tab:features} describe dichas caracter\'isticas.


\begin{table}[h]
	\centering
	\caption{Caracter\'isticas usadas como entrada para el clasificador de SVM}
	\label{tab:features}
	\begin{tabular}{|l|l|l|l|l|}
		Se\~nal & Caracter\'istica & Descripci\'on & Unidad \\
		GSR&\textit{Valor en el pico}                &             &	$\mu S$ \\
		GSR   &\textit{Amplitud del pico}                &             &$\mu S$ 	\\
		GSR   &\textit{Valor en el punto de crecimiento}                &             &$\mu S$	 \\
		GSR   &\textit{Indice de media recuperac\'on}                &             &Segundos	 \\
		GSR   &\textit{Valor de media recuperac\'on}                &             &$\mu S$	 \\
		GSR   &\textit{Distancia el pico anterior}                &             &Segundos	 \\
		HR   &\textit{M\'aximo}                &             &BPM	 \\
		HR   &\textit{Promedio}                &             &BPM	 \\
		IBI   &\textit{M\'inimo}                &             &Segundos	 \\
		IBI   &\textit{Promedio}                &             &Segundos	 \\
		IBI   &\textit{Desviaci\'on estandar}                &             &Segundos	 \\
		TEMP   &\textit{M\'aximo}                &             &$\degree C$	 \\
		TEMP   &\textit{Promedio}                &             &$\degree C$ \\

		
	\end{tabular}
\end{table}
\section{Resultados de aprendizaje de m\'aquina}
\newpage
%%=====================================================
} }%
%BeginExpansion

\chapter{Evaluacion}\label{capit:cap4}
\vspace{-2.0325ex}%
\noindent
\rule{\textwidth}{0.5pt}
\vspace{-5.5ex}% 
\newcommand{\pushline}{\Indp}% Indent puede ir o no :p

\section{Introducci\'on}\label{cap4:intro}

\section{Codificaci\'on}

\section{Extracci\'on de caracter\'isticas}
Por cada tipo de se\~nal, se tomaron varias caracter\'isticas que generalizan un segmento de datos. La tabla ~\ref{tab:features} describe dichas caracter\'isticas.


\begin{table}[h]
	\centering
	\caption{Caracter\'isticas usadas como entrada para el clasificador de SVM}
	\label{tab:features}
	\begin{tabular}{|l|l|l|l|l|}
		Se\~nal & Caracter\'istica & Descripci\'on & Unidad \\
		GSR&\textit{Valor en el pico}                &             &	$\mu S$ \\
		GSR   &\textit{Amplitud del pico}                &             &$\mu S$ 	\\
		GSR   &\textit{Valor en el punto de crecimiento}                &             &$\mu S$	 \\
		GSR   &\textit{Indice de media recuperac\'on}                &             &Segundos	 \\
		GSR   &\textit{Valor de media recuperac\'on}                &             &$\mu S$	 \\
		GSR   &\textit{Distancia el pico anterior}                &             &Segundos	 \\
		HR   &\textit{M\'aximo}                &             &BPM	 \\
		HR   &\textit{Promedio}                &             &BPM	 \\
		IBI   &\textit{M\'inimo}                &             &Segundos	 \\
		IBI   &\textit{Promedio}                &             &Segundos	 \\
		IBI   &\textit{Desviaci\'on estandar}                &             &Segundos	 \\
		TEMP   &\textit{M\'aximo}                &             &$\degree C$	 \\
		TEMP   &\textit{Promedio}                &             &$\degree C$ \\

		
	\end{tabular}
\end{table}
\section{Resultados de aprendizaje de m\'aquina}
\newpage
%%=====================================================

%EndExpansion
\newpage }

{\normalsize
%TCIMACRO{\QSubDoc{Include Capitulo05}{
\chapter{Conclusiones y trabajo a futuro}\label{capit:cap5}
\vspace{-2.0325ex}%
\noindent
\rule{\textwidth}{0.5pt}
\vspace{-5.5ex}% 
\newcommand{\pushline}{\Indp}% Indent puede ir o no :p

\section{Introducci\'on}
	Existe una tendencia mundial hacia el envejecimiento de la población. El incremento en el n\'umero de adultos mayores obedece principalmente a mejoras en el sistema de salud y una menor tasa de natalidad. En países en desarrollo, como M\'exico, esta fenómeno se está desarrollando de manera más acelerada. El papel del cuidador se vuelve un elemento importante en la salud del adulto mayor. La demanda del cuidador para cubrir las necesidades del adulto mayor son muy variadas y complejas y pueden desecadenar problemas como la depresi\'on y la ansiedad.
En este estudio, se desarroll\'o un estudio de usuario con situaciones simuladas parecidas a las reales a trav\'es de una t\'ecnica llamada ``Naturalist Enactment'' resultando en una base de datos fisiol\'ogicos de 10 personas diferentes bajo situaciones de ansiedad. Se describi\'o como se obtuvo ``Ground truth'' y el procesamiento realizado a las se\~nales fisiol\'ogicas de diferentes dispositivos vestibles. Por \'ultimo, se mostr\'o que el uso de c\'omputo vestible para la detecci\'on de ansiedad en situaciones naturalistas es posible.

Los m\'etodos de investigaci\'on expuestos y los resultados preliminares obtenidos muestran que el c\'omputo vestible podr\'ia ser utilizado para la detecci\'on de ansiedad. Sin embargo, hacen falta m\'as estudios para resolver el problema completamente.


\section{Contribuci\'on}
	Como resultados de este estudio, se pueden indicar las siguientes contribuciones:

	\begin{itemize}
		\item \textbf{Estudio de usuario simulado:} Una de las tendencias actuales en las ciencias de la computaci\'on es realizar estudio de usuarios fuera del laboratorio. Este ambiente con menos controles, ruidosos, impredecibles y dif\'iciles de analizar permiten simular situaciones m\'as parecidas a la vida real. En este estudio se mostr\'o como se pueden realizar estudio de usuarios en situaciones simuladas obtener resultados cuantificables.

	\item \textbf{Base de datos:} Se gener\'o una base de datos fisiol\'ogicas de personas bajo condiciones parecidas a las de los cuidadores de personas con demencia. Con esta base de datos es posible hacer pruebas de nuevos m\'etodos de detecci\'on de ansiedad sin necesidad de realizar un estudio de usuario nuevo. La base de datos abre tambi\'en la posibilidad de analizar aspectos no considerados originalmente de las reacciones de los individuos bajo las condiciones de estr\'es con los datos de EEG.

	\item \textbf{An\'alisis preliminar:} Se hizo un an\'alisis preliminar de los datos con resultados promisorios que dan evidencia de que es factible hacer detecci\'on por medio de c\'omputo vestible.
	\item \textbf{Art\'iculo de congreso:} Se escribi\'o el art\'iculo \textit{``Detecting State Anxiety when Careing for People with Dementia. Miranda D., Jesus F., Catalina I.''}. Aceptado para publicaci\'on en ``AmiHEALTH 2015, International Conference on Ambient Intelligence and Health, Diciembre 2015''
	\end{itemize}
\section{Limitaciones}
        Un problema es que se obtuvieron muy pocas muestras de ansiedad por participante, una intervenci\'on mas largar solucionar\'ia este probema. Sin embargo, aumentar el n\'umero de sesiones y/o participantes aumentari\'ia el tiempo necesario para obtener la l\'inea base.

	Otra dificultad es obtener informaci\'on de l\'inea base. A pesar de que se obtuvo informaci\'on \'util con el m\'etodo de etiquetado, se necesitaron cerca de 8 horas por sesi\'on para completarlos. En general, se requieren maneras confiables no exhaustivas para las aplicaciones conscientes de la ansiedad.

	En este estudio participaron personas que no eran cuidadores en el momento de las pruebas. Los cuidadores reales tienen necesidades complejas y su sensibilidad a los est\'imulos de la persona a quien cuidan podr\'ian ser diferentes.
\section{Trabajo a futuro}
Algunas consideraciones de trabajo a futuro son las siguientes:

\begin{itemize}
	\item \textbf{Obtenci\'on de l\'inea base:} El proceso actual de etiquetado de segmentos de ansiedad requiere de mucho tiempo y trabajo humano. Una opci\'on para reducir el tiempo es utilizar la informaci\'on de EEG capturada.
	\item \textbf{Algortimos de clasificaci\'on:} Las pruebas realizadas a los datos son preliminares. Se deben de considerar t\'ecnicas diferentes que se adapten mejor a estos tipos de situaciones y que reporten mayor precisi\'on y exhaustividad.
	\item \textbf{Utilizar datos de los participantes con experiencia de cuidador:} Los datos utilizados corresponden a lo participantes sin experiencia de cuidador. Un estudio para analizar la diferencia entre los participantes con y sin experiencia ser\'ia un paso mas para acercarse a resolver el problema de ansiedad en cuidadores.
	\item \textbf{Aplicaci\'on consciente de la ansiedad:} Realizar una aplicaci\'on en dispositivos m\'oviles y/o vestibles que incluya las ``estrategias de afrontamiento''
\end{itemize}
%%=====================================================
} }%
%BeginExpansion

\chapter{Conclusiones y trabajo a futuro}\label{capit:cap5}
\vspace{-2.0325ex}%
\noindent
\rule{\textwidth}{0.5pt}
\vspace{-5.5ex}% 
\newcommand{\pushline}{\Indp}% Indent puede ir o no :p

\section{Introducci\'on}
	Existe una tendencia mundial hacia el envejecimiento de la población. El incremento en el n\'umero de adultos mayores obedece principalmente a mejoras en el sistema de salud y una menor tasa de natalidad. En países en desarrollo, como M\'exico, esta fenómeno se está desarrollando de manera más acelerada. El papel del cuidador se vuelve un elemento importante en la salud del adulto mayor. La demanda del cuidador para cubrir las necesidades del adulto mayor son muy variadas y complejas y pueden desecadenar problemas como la depresi\'on y la ansiedad.
En este estudio, se desarroll\'o un estudio de usuario con situaciones simuladas parecidas a las reales a trav\'es de una t\'ecnica llamada ``Naturalist Enactment'' resultando en una base de datos fisiol\'ogicos de 10 personas diferentes bajo situaciones de ansiedad. Se describi\'o como se obtuvo ``Ground truth'' y el procesamiento realizado a las se\~nales fisiol\'ogicas de diferentes dispositivos vestibles. Por \'ultimo, se mostr\'o que el uso de c\'omputo vestible para la detecci\'on de ansiedad en situaciones naturalistas es posible.

Los m\'etodos de investigaci\'on expuestos y los resultados preliminares obtenidos muestran que el c\'omputo vestible podr\'ia ser utilizado para la detecci\'on de ansiedad. Sin embargo, hacen falta m\'as estudios para resolver el problema completamente.


\section{Contribuci\'on}
	Como resultados de este estudio, se pueden indicar las siguientes contribuciones:

	\begin{itemize}
		\item \textbf{Estudio de usuario simulado:} Una de las tendencias actuales en las ciencias de la computaci\'on es realizar estudio de usuarios fuera del laboratorio. Este ambiente con menos controles, ruidosos, impredecibles y dif\'iciles de analizar permiten simular situaciones m\'as parecidas a la vida real. En este estudio se mostr\'o como se pueden realizar estudio de usuarios en situaciones simuladas obtener resultados cuantificables.

	\item \textbf{Base de datos:} Se gener\'o una base de datos fisiol\'ogicas de personas bajo condiciones parecidas a las de los cuidadores de personas con demencia. Con esta base de datos es posible hacer pruebas de nuevos m\'etodos de detecci\'on de ansiedad sin necesidad de realizar un estudio de usuario nuevo. La base de datos abre tambi\'en la posibilidad de analizar aspectos no considerados originalmente de las reacciones de los individuos bajo las condiciones de estr\'es con los datos de EEG.

	\item \textbf{An\'alisis preliminar:} Se hizo un an\'alisis preliminar de los datos con resultados promisorios que dan evidencia de que es factible hacer detecci\'on por medio de c\'omputo vestible.
	\item \textbf{Art\'iculo de congreso:} Se escribi\'o el art\'iculo \textit{``Detecting State Anxiety when Careing for People with Dementia. Miranda D., Jesus F., Catalina I.''}. Aceptado para publicaci\'on en ``AmiHEALTH 2015, International Conference on Ambient Intelligence and Health, Diciembre 2015''
	\end{itemize}
\section{Limitaciones}
        Un problema es que se obtuvieron muy pocas muestras de ansiedad por participante, una intervenci\'on mas largar solucionar\'ia este probema. Sin embargo, aumentar el n\'umero de sesiones y/o participantes aumentari\'ia el tiempo necesario para obtener la l\'inea base.

	Otra dificultad es obtener informaci\'on de l\'inea base. A pesar de que se obtuvo informaci\'on \'util con el m\'etodo de etiquetado, se necesitaron cerca de 8 horas por sesi\'on para completarlos. En general, se requieren maneras confiables no exhaustivas para las aplicaciones conscientes de la ansiedad.

	En este estudio participaron personas que no eran cuidadores en el momento de las pruebas. Los cuidadores reales tienen necesidades complejas y su sensibilidad a los est\'imulos de la persona a quien cuidan podr\'ian ser diferentes.
\section{Trabajo a futuro}
Algunas consideraciones de trabajo a futuro son las siguientes:

\begin{itemize}
	\item \textbf{Obtenci\'on de l\'inea base:} El proceso actual de etiquetado de segmentos de ansiedad requiere de mucho tiempo y trabajo humano. Una opci\'on para reducir el tiempo es utilizar la informaci\'on de EEG capturada.
	\item \textbf{Algortimos de clasificaci\'on:} Las pruebas realizadas a los datos son preliminares. Se deben de considerar t\'ecnicas diferentes que se adapten mejor a estos tipos de situaciones y que reporten mayor precisi\'on y exhaustividad.
	\item \textbf{Utilizar datos de los participantes con experiencia de cuidador:} Los datos utilizados corresponden a lo participantes sin experiencia de cuidador. Un estudio para analizar la diferencia entre los participantes con y sin experiencia ser\'ia un paso mas para acercarse a resolver el problema de ansiedad en cuidadores.
	\item \textbf{Aplicaci\'on consciente de la ansiedad:} Realizar una aplicaci\'on en dispositivos m\'oviles y/o vestibles que incluya las ``estrategias de afrontamiento''
\end{itemize}
%%=====================================================

%EndExpansion
\newpage }


\linespread{1.0}
\addcontentsline{toc}{chapter}{\normalsize\expandafter{Lista de referencias}}
{\normalsize
\bibliographystyle{cicese}%cicese, ieeetr, plain, unsrt, alpha, abbrv, acm, apalike 
\nocite{*}
\bibliography{iTesis}
}

\linespread{1.5}
{\normalsize
%TCIMACRO{\QSubDoc{Include Apendice}{\input{ApendiceA.tex}} }%
%BeginExpansion
\appendix{}

\chapter{Carta de consentimiento informado de participantes en el proyecto de investigaci\'on} \label{aped:A}
\vspace{-3ex}%
\noindent
\rule{\textwidth}{1pt}
\vspace{-2ex}%

Por medio de la presente acepto participar en el proyecto de investigación que CICESE realiza, titulado: \textbf{``Monitoreo por sensores electrónicos en adultos mayores sanos en Ensenada, B.C,''}, a llevarse a cabo en esta ciudad de Mayo a Septiembre de 2015.

\textbf{El objetivo de este estudio es:} Obtener información capturada a través de teléfonos celulares  inteligentes, de las actividades cotidianas que realizan los adultos mayores de 65 años de edad.

\textbf{Se me ha explicado que mi participación consistirá en:}  Llevar a cabo sesiones de terapias a adultos mayores con Alzheimer de alrededor de una hora de duración. También se me informó que debo de contestar a diversas preguntas sobre mi desempeño en la terapia en forma de cuestionarios. Por último, fui informado que recibiré una compensación de un pase doble al cine en la pelicula disponible de mi elección al termino de la sesión.

\textbf{Se me ha informado ampliamente sobre los posibles riesgos, inconvenientes, molestias y beneficios derivados de la participación en el estudio.}

Podré dejar de contestar cualquier pregunta  si no es mi deseo, o si dudo de la respuesta.
El investigador principal se ha comprometido a responder cualquier pregunta y aclarar cualquier duda que tenga, acerca de los procedimientos que se llevarán a cabo, o cualquier otro asunto relacionado con la investigación.
Entiendo que conservo el derecho a retirarme del estudio en cualquier momento en que lo considere conveniente.
Los investigadores a cargo del proyecto me ha asegurado de que no se me identificará en las presentaciones o publicaciones que se deriven de este estudio, y que los datos recabados  serán manejados en manera confidencial. Al mismo tiempo se me ha asegurado, que si así lo deseo, se me proporcionará la información que se derive del estudio.

Nombre y firma del participante

Domicilio del participante

\appendix{}

\chapter{Carta de no divulgaci\'on de participantes en el proyecto de investigaci\'on} \label{aped:B}
\vspace{-3ex}%
\noindent
\rule{\textwidth}{1pt}
\vspace{-2ex}%

El siguiente documento describe la manera en que los participantes deberán de hacer uso de la información generada durante las actividades del estudio con el fin de asegurar la calidad de los datos obtenidos.

Definiciones:
\begin{itemize}
	\item \textbf{Participante:} Persona reclutada para realizar las actividades durante el estudio.
	\item \textbf{Adulto Mayor:} Persona reclutada con la cual el participante realizará las actividades durante el estudio.
	\item \textbf{Investigadores:} Personal del CICESE (estudiante de posgrado, profesor o auxiliar de investigador) responsable del estudio.
\end{itemize}
Dentro de las limitaciones, el participante no podrá:
\begin{itemize}
	\item Comentar la situación mental o física del adulto mayor. 
	\item Describir las actividades que realizó durante las terapias. 
	\item Compartir técnicas o cualquier otra información que pueda ser utilizada para afectar el resultado de las terapias de otros participantes.
\end{itemize}

Todas las inquietudes o dudas deben de ser canalizadas a través de los investigadores por las formas de contacto proporcionadas.

Todas estas limitaciones son vigentes durante el tiempo de la intervención. Una vez que todos los participantes hayan terminado su participación este acuerdo será invalidado.

Nombre y firma del participante

Domicilio del participante

%EndExpansion
\newpage }



\end{document}
