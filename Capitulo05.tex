
\chapter{Conclusiones y trabajo a futuro}\label{capit:cap5}
\vspace{-2.0325ex}%
\noindent
\rule{\textwidth}{0.5pt}
\vspace{-5.5ex}% 
\newcommand{\pushline}{\Indp}% Indent puede ir o no :p

\section{Introducci\'on}
	Existe una tendencia mundial hacia el envejecimiento de la población. El incremento en el n\'umero de adultos mayores obedece principalmente a mejoras en el sistema de salud y una menor tasa de natalidad. En países en desarrollo, como M\'exico, esta fenómeno se está desarrollando de manera más acelerada. El papel del cuidador se vuelve un elemento importante en la salud del adulto mayor. La demanda del cuidador para cubrir las necesidades del adulto mayor son muy variadas y complejas y pueden desecadenar problemas como la depresi\'on y la ansiedad.
En este estudio, se desarroll\'o un estudio de usuario con situaciones simuladas parecidas a las reales a trav\'es de una t\'ecnica llamada ``Naturalist Enactment'' resultando en una base de datos fisiol\'ogicos de 10 personas diferentes bajo situaciones de ansiedad. Se describi\'o como se obtuvo ``Ground truth'' y el procesamiento realizado a las se\~nales fisiol\'ogicas de diferentes dispositivos vestibles. Por \'ultimo, se mostr\'o que el uso de c\'omputo vestible para la detecci\'on de ansiedad en situaciones naturalistas es posible.

Los m\'etodos de investigaci\'on expuestos y los resultados preliminares obtenidos muestran que el c\'omputo vestible podr\'ia ser utilizado para la detecci\'on de ansiedad. Sin embargo, hacen falta m\'as estudios para resolver el problema completamente.


\section{Contribuci\'on}
	Como resultados de este estudio, se pueden indicar las siguientes contribuciones:

	\begin{itemize}
		\item \textbf{Estudio de usuario simulado:} Una de las tendencias actuales en las ciencias de la computaci\'on es realizar estudio de usuarios fuera del laboratorio. Este ambiente con menos controles, ruidosos, impredecibles y dif\'iciles de analizar permiten simular situaciones m\'as parecidas a la vida real. En este estudio se mostr\'o como se pueden realizar estudio de usuarios en situaciones simuladas obtener resultados cuantificables.

	\item \textbf{Base de datos:} Se gener\'o una base de datos fisiol\'ogicas de personas bajo condiciones parecidas a las de los cuidadores de personas con demencia. Con esta base de datos es posible hacer pruebas de nuevos m\'etodos de detecci\'on de ansiedad sin necesidad de realizar un estudio de usuario nuevo. La base de datos abre tambi\'en la posibilidad de analizar aspectos no considerados originalmente de las reacciones de los individuos bajo las condiciones de estr\'es con los datos de EEG.

	\item \textbf{An\'alisis preliminar:} Se hizo un an\'alisis preliminar de los datos con resultados promisorios que dan evidencia de que es factible hacer detecci\'on por medio de c\'omputo vestible.
	\item \textbf{Art\'iculo de congreso:} Se escribi\'o el art\'iculo \textit{``Detecting State Anxiety when Careing for People with Dementia. Miranda D., Jesus F., Catalina I.''}. Aceptado para publicaci\'on en ``AmiHEALTH 2015, International Conference on Ambient Intelligence and Health, Diciembre 2015''
	\end{itemize}
\section{Limitaciones}
        Un problema es que se obtuvieron muy pocas muestras de ansiedad por participante, una intervenci\'on mas largar solucionar\'ia este probema. Sin embargo, aumentar el n\'umero de sesiones y/o participantes aumentari\'ia el tiempo necesario para obtener la l\'inea base.

	Otra dificultad es obtener informaci\'on de l\'inea base. A pesar de que se obtuvo informaci\'on \'util con el m\'etodo de etiquetado, se necesitaron cerca de 8 horas por sesi\'on para completarlos. En general, se requieren maneras confiables no exhaustivas para las aplicaciones conscientes de la ansiedad.

	En este estudio participaron personas que no eran cuidadores en el momento de las pruebas. Los cuidadores reales tienen necesidades complejas y su sensibilidad a los est\'imulos de la persona a quien cuidan podr\'ian ser diferentes.
\section{Trabajo a futuro}
Algunas consideraciones de trabajo a futuro son las siguientes:

\begin{itemize}
	\item \textbf{Obtenci\'on de l\'inea base:} El proceso actual de etiquetado de segmentos de ansiedad requiere de mucho tiempo y trabajo humano. Una opci\'on para reducir el tiempo es utilizar la informaci\'on de EEG capturada.
	\item \textbf{Algortimos de clasificaci\'on:} Las pruebas realizadas a los datos son preliminares. Se deben de considerar t\'ecnicas diferentes que se adapten mejor a estos tipos de situaciones y que reporten mayor precisi\'on y exhaustividad.
	\item \textbf{Utilizar datos de los participantes con experiencia de cuidador:} Los datos utilizados corresponden a lo participantes sin experiencia de cuidador. Un estudio para analizar la diferencia entre los participantes con y sin experiencia ser\'ia un paso mas para acercarse a resolver el problema de ansiedad en cuidadores.
	\item \textbf{Aplicaci\'on consciente de la ansiedad:} Realizar una aplicaci\'on en dispositivos m\'oviles y/o vestibles que incluya las ``estrategias de afrontamiento''
\end{itemize}
%%=====================================================
