%TCIDATA{Version=5.00.0.2552}
%TCIDATA{LaTeXparent=0,0,Tesis.tex}
Debido a la tendencia en el aumento del n\'umero de adultos mayores con problemas de demencia, la cantidad de cuidadores ha ido en aumento. La ansiedad es uno de los problemas que mas afecta a los cuidadores. Los estudios actuales para medir la ansiedad se basan en t\'ecnicas tradicionales como encuestas o cuestionarios y solo muestran resultados al largo plazo y miden el nivel de ansiedad experimentada en los \'ultimos d\'ias o semanas. El c\'omputo vestible (computadoras lo suficientemente peque\~nas para ser llevadas como ropa) tiene la capacidad de sensar al usuario constantemente, por lo que da la oportunidad de detectar periodos de ansiedad en el momento. A pesar de que los efectos de la ansiedad se muestran a largo plazo, los cuidadores podr\'ian ser beneficiados de la detecci\'on de la ansiedad situacional. En este trabajo, se reporta el desarrollo de un experimento para recabar datos de cuidadores de personas con demencia en situaciones naturalistas y se muestra una t\'ecnica para la detecci\'on de ansiedad. Participaron 10 sujetos en 3 sesiones de terapias cognitivas con un persona que actu\'o como si tuviera demencia. Se obtuvo una base de datos fisiol\'ogicos como resistencia galv\'anica de la piel, ritmo cardi\'aco, datos de acelerometr\'ia y electroencefalograma, entre otros. Se presenta un estudio preliminar que muestra como el c\'omputo vestible es apto para detectar periodos de ansiedad en cuidadores de personas con demencia.
