%TCIDATA{Version=5.00.0.2552}
%TCIDATA{LaTeXparent=0,0,Tesis.tex}
La ansiedad es uno de los problemas que m\'as afecta a los cuidadores de personas con demencia. Los estudios actuales para medir la ansiedad se basan en t\'ecnicas tradicionales como encuestas o cuestionarios y solo muestran resultados a largo plazo y miden el nivel de ansiedad experimentada en los \'ultimos d\'ias o semanas. El c\'omputo vestible (computadoras lo suficientemente peque\~nas para ser llevadas como ropa) tiene la capacidad de sensar al usuario constantemente, por lo que da la oportunidad de detectar periodos de ansiedad en el momento en que el individuo los experimenta. A pesar de que los efectos de la ansiedad se muestran a largo plazo, los cuidadores podr\'ian ser beneficiados de la detecci\'on de la ansiedad situacional. En este trabajo, se reporta el desarrollo de un experimento para recabar datos de cuidadores de personas con demencia en situaciones simuladas. En este estudio, 10 personas participaron en 3 sesiones de terapias cognitivas con un persona que actu\'o como si tuviera demencia. A los participantes se les dijo que el actor era una persona que realmente ten\'ia problemas de demencia. Se obtuvo una base de datos fisiol\'ogicos de 974.64 minutos de duraci\'on. Esta base de datos contiene datos de resistencia galv\'anica de la piel, ritmo cardi\'aco, datos de acelerometr\'ia y electroencefalograma, entre otros. Se presentan resultados preliminares que muestran como el c\'omputo vestible es apto para detectar periodos de ansiedad en cuidadores de personas con demencia.
