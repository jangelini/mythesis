
\chapter{Detecci\'on de ansiedad y resultados}\label{capit:cap4}
\vspace{-2.0325ex}%
\noindent
\rule{\textwidth}{0.5pt}
\vspace{-5.5ex}% 
\newcommand{\pushline}{\Indp}% Indent puede ir o no :p

\section{Introducci\'on}\label{cap4:intro}
En este cap\'itulo se describe como se utilizaron los datos recabados con el experimento. Se muestran las pruebas realizadas a los datos recabados en el experimento y los rultados de la detecci\'on por medio del uso de m\'aquinas de soporte vectorial.

%El archivo resultante se export\'o a Google Spreedsheets (Ver Figura: ~\ref{fig:img_codification} ) para hacer el etiquetado de segmentos de ansiedad. Esta codificaci\'on se hizo en base a la tabla de codificaci\'on (Ver tabla: ~\ref{table:anxilevels}). Se utilizaron las notas de observaci\'on recabadas para comparar que los eventos codificados a partir del video coincidieran en el tiempo y nivel de evento. Debido a que existi\'o un retraso del momento en que se inici\'o a grabar los datos fisiol\'ogicos y el momento en el que se inici\'o la grabaci\'on, se tuvieron que sincronizar los datos de la codificaci\'on. Esto se logr\'o grabando ante la c\'amara el tiempo que mostraba el tel\'efono m\'ovil encargado de la colecci\'on de los datos. Por medio de un herramienta, se sincronizaron los tiempos de la transcripci\'on en base a los segundos reportados en el video y se convirtieron de segundos a tiempo unix. La precisi\'on lograda con esta t\'ecnica fu\'e de 0 a 1 segundos.


%Esto permiti\'o realizar la segmentaci\'on de los eventos para su posterior an\'alisis. Tambi\'en da una visi\'on general del nivel de ansiedad observado durante toda la sesi\'on en comparaci\'on con los datos fisiol\'ogicos (Ver figura: ~\ref{fig:session_anxietylevel}).


\section{Resultados de aprendizaje de m\'aquina}
Se realizaron diferentes pruebas con diferentes ``kernels'' de la m\'aquina de soporte vectorial. En todas las pruebas excepto X se utiliz\'o una partici\'on del 80\% como entrenamiento y 20\% como prueba
\subsection{Pruebas por sujeto}
\subsection{Pruebas entre sujetos}
\subsection{Prueba con conjunto}
\subsection{Prueba por sesi\'on}


\newpage
%%=====================================================
