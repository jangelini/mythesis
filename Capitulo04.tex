
\chapter{Evaluacion}\label{capit:cap4}
\vspace{-2.0325ex}%
\noindent
\rule{\textwidth}{0.5pt}
\vspace{-5.5ex}% 
\newcommand{\pushline}{\Indp}% Indent puede ir o no :p

\section{Introducci\'on}\label{cap4:intro}

\section{Codificaci\'on}

\section{Extracci\'on de caracter\'isticas}
Por cada tipo de se\~nal, se tomaron varias caracter\'isticas que generalizan un segmento de datos. La tabla ~\ref{tab:features} describe dichas caracter\'isticas.


\begin{table}[h]
	\centering
	\caption{Caracter\'isticas usadas como entrada para el clasificador de SVM}
	\label{tab:features}
	\begin{tabular}{|l|l|l|l|l|}
		Se\~nal & Caracter\'istica & Descripci\'on & Unidad \\
		GSR&\textit{Valor en el pico}                &             &	$\mu S$ \\
		GSR   &\textit{Amplitud del pico}                &             &$\mu S$ 	\\
		GSR   &\textit{Valor en el punto de crecimiento}                &             &$\mu S$	 \\
		GSR   &\textit{Indice de media recuperac\'on}                &             &Segundos	 \\
		GSR   &\textit{Valor de media recuperac\'on}                &             &$\mu S$	 \\
		GSR   &\textit{Distancia el pico anterior}                &             &Segundos	 \\
		HR   &\textit{M\'aximo}                &             &BPM	 \\
		HR   &\textit{Promedio}                &             &BPM	 \\
		IBI   &\textit{M\'inimo}                &             &Segundos	 \\
		IBI   &\textit{Promedio}                &             &Segundos	 \\
		IBI   &\textit{Desviaci\'on estandar}                &             &Segundos	 \\
		TEMP   &\textit{M\'aximo}                &             &$\degree C$	 \\
		TEMP   &\textit{Promedio}                &             &$\degree C$ \\

		
	\end{tabular}
\end{table}
\section{Resultados de aprendizaje de m\'aquina}
\newpage
%%=====================================================
