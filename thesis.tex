\documentclass[letterpaper,12pt]{cicese}
	\usepackage{natbib}        
	\usepackage[letterpaper,left=2.5cm,right=2.5cm,top=3cm,bottom=3cm]{geometry}
        \usepackage{setspace}
        \usepackage{natbib}
	\usepackage{cicese}
	\usepackage{hyperref}
	\hypersetup{pdftex,colorlinks=false,allcolors=blue}
	\usepackage{hypcap}
	\input{tcilatex}
	\begin{document}
	\doublespace
	\title{Sensado Oportunista para la detecci\'on de ansiedad en cuidadores de personas que sufren de discacapidades mentales}
	\author{Dari\'en Alberto Miranda Boj\'orquez}
	\maketitle
        \maketitle
        \newpage
        \tableofcontents
        \newpage


                \chapter{Introducci\'on}
                        La ansiedad es una emoci\'on caracterizada por sensaciones de tensi\'on, pensamientos de preocupaci\'on y cambios f\'isicos como incremento en la presi\'on arterial.\citep{psychologyapa}.Una forma com\'un en donde la ansiedad se presenta es en el estr\'es. La relaci\'on entre el estr\'es y la ansiedad es la ansiedad es la se\~nal psicofisiol\'ogica de que la respuesta al estr\'es ha sido iniciada \citep{PMID2235645}.Es com\'un que la poblaci\'on en general tenga episodios de ansiedad debido a los tipos de trabajo de nuestra sociedad moderna. Durante esos lapsos de tiempo, la persona suele experimentar un nivel de ansiedad el cual es una reacci\'on normal para lograr objetivos. Sin embargo, cuando la persona experimenta un nivel de ansiedad el cual es tan alto que no le permite manejar su vida normal, se dice que la persona tiene un desorden de ansiedad\citep{repetto2013}.

                        Uno de los sectores de poblaci\'on vulnerables, son los cuidadores de pacientes con autismo o demencia. Se encuentra documentado que los cuidadores, al llevar una carga f\'isica, cognitiva y emocional derivada de su labor les genera padecimientos como ansiedad, estr\'es, y hasta la muerte\citep{Chen2013}. Debido a que los cuidadores no necesariamente son personas con una formaci\'on profesional, estos efectos pueden verse aumentados. Por lo general, los cuidadores que son familiares del paciente son a\'un m\'as afectados ya que necesitan administrar el tiempo de trabajo, familia, actividades sociales y la actividad misma del cuidado del paciente.
			%Cuales son las situaciones ( escenarios ) en los que los CUIDADORES presentan ansiedad?
                        %autismo
                        El desorden de autismo (referido como autismo) es una de las variedades del Espectro de Des\'ordenes del Autismo (ASD) y est\'a caracterizado por interacciones sociales da\~nadas, ausencia de habilidades de comunicaci\'on, movimientos estereotipados y mal comportamiento en general\citep{bernier2010autism}. Existen escuelas especiales para ni\~nos con este padecimiento.

                        %demencia
La demencia es un s\'indrome del declive de las habilidades cognitivas. Los s\'intomas comunes son: problemas de memoria, dificultades para realizar tareas familiares, mal juicio, deterioro del lenguaje hablado y cambios de humor\citep{Aziz}. Afecta alrededor de el 4\% de las personas mayores de 65 a\~nos y al 40\% de las personas mayores de 90.

                        Por otro lado, el c\'omputo vestible nos permite llevar computadoras con nosotros de la misma manera que llevamos la ropa puesta. Al ``vestir" un dispositivo,
                        el usuario tiene acceso a una computadora que es capaz de monitorearlo a \'el y a su entorno por medio de sensores. Los sensores pueden medir entre
                        otras cosas: movimientos del cuerpo del usuario, la posici\'on del usuario, intensidad de luz, ruido, im\'agenes de su ambiente, ritmo card\'iaco, capacidad
                        conductiva de la piel, distancias, entre otros. Debido a su caracter\'istica de ser vestible, se pueden hacer monitoreos constantes y mas precisos que con
                        los sistemas tradicionales y ayudar en las tareas de la vida cotidiana.

                        El uso de c\'omputo vestible para la detecci\'on de la ansiedad abre una posibilidad para ayudar a reducir el riesgo a la salud mental
                        de los cuidadores de pacientes con autismo o demencia. La siguiente secci\'on ejemplifica posibles escenarios de aplicaciones reales

	\chapter{Fundamentos te\'oricos}
	\section{Qu\'e es la ansiedad?}
			La ansiedad es bla bla bla
			\section{Cuidadores}
		
			\section{Tipos de cargas en los cuidadores}


	\newpage
		\chapter{Mis notas}
			Esta secci\'on no es parte de la tesis. Aqu\'i agrego las citas que considero interesantes para mi tesis y que a\'un no encuentro el lugar adecuado donde ponerlas.

	
	
		En un estudio, 60\% de los cuidadores desarrollaron un desorden depresivo y/o de ansiedad en los primeros 24 meses: 37\% de depresi\'on, 55\% desorden de ansiedad y 32\% ambos. \citep{Joling2014}


		Mientras que la carga de los cuidadores informales aumenta, se vuelve mas probable que sufran de ansiedad y depresion \citep{Denno20131731}

		La carga en los cuidadores (f\'isica o psicol\'ogica) podr\'ia aumentar los niveles de ansiedad. Entre mas demandante es un servicio, mayor podr\'ia ser la ansiedad percibida. Los comportamientos bizarros o impredecibles de la persona afectada por demencia aumentan la carga emocional \citep{Rosa201054}

		Induciendo ansiedad: Una manera que se puede inducir ansiedad es por medio de situciones generadas. Podriamos poner a una persona en una situacion virtual de ansiedad mientras que sus signos
		vitales son monitorizados. Al dar una retroalimentacion biologica, hacemos consciente al usuario de tal manera que la ansiedad sea inducida.

		Como medirlo: Por medio de la disminucion de la ansiedad y la revision del bienestar percibido de la persona
		
		Los aspectos cognitivos se encuentran conectados a los sentimientos de la persona \citep{Wilt2011987}. Por lo que su estado emocional puede afectar en la percepci\'on de la dificultad de la tarea a realizar.

		H1: \textit{Los sujetos con un nivel de ansiedad mayor encontrar\'an mas dif\'icil de hacer una tarea altamente demandante que aquellos con un nivel de ansiedad bajo}.
		

		H2: \textit{Los sujetos con un nivel de ansiedad mayor tendr\'an valores de HR, SBR ( twitching ) y  de ST aumentar\'an durante una tarea altamente demandante}.
		\newpage
	
		Algunas de las actividades que realizan los cuidadores de personas con demencia son las siguientes \citep{tagkey2008110}: 
		\begin{itemize}
			\item Comprar alimentos, preparar comidas y proveer transporte.
			\item Ayudar a la persona a tomar medicinas correctamente y seguir las recomendaciones de los tratamientos para su demencia y otras condiciones m\'edicas.
			\item Administrar las financias y asuntos legales.
			\item Supervisar a la persona para evitar actividades peligrosas, como deambular y perderse.	
			\item Ba\~nar, vestir, alimentar y ayudar a la persona a usar el ba\~no o proveer cuidados para la incontienencia.
			\item Hacer citas para cuidados m\'edicos.
			\item Lidiar con s\'intomas de comportamiento.

		\end{itemize}
        {\normalsize
		\bibliographystyle{cicese}
                \bibliography{thesis}
        }

\end{document}
