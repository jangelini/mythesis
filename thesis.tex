\documentclass{report}
	\usepackage{natbib}        
	\begin{document}
	

En un estudio, 60\% de los cuidadores desarrollaron un desorden depresivo y/o de ansiedad en los primeros 24 meses: 37\% de depresi\'on, 55\% desorden de ansiedad y 32\% ambos. \citep{Joling2014}


Mientras que la carga de los cuidadores informales aumenta, se vuelve mas probable que sufran de ansiedad y depresion \citep{Denno20131731}

Induciendo ansiedad: Una manera que se puede inducir ansiedad es por medio de situciones generadas. Podriamos poner a una persona en una situacion virtual de ansiedad mientras que sus signos
vitales son monitorizados. Al dar una retroalimentacion biologica, hacemos consciente al usuario de tal manera que la ansiedad sea inducida.
        {\normalsize
		\bibliographystyle{cicese}
                \bibliography{thesis}
        }

\end{document}
