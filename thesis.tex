\documentclass[letterpaper,12pt]{cicese}
	\usepackage{natbib}        
	\usepackage[letterpaper,left=2.5cm,right=2.5cm,top=3cm,bottom=3cm]{geometry}
        \usepackage{setspace}
        \usepackage{natbib}
	\usepackage{cicese}
	\usepackage{hyperref}
	\hypersetup{pdftex,colorlinks=false,allcolors=blue}
	\usepackage{hypcap}
	\input{tcilatex}
	\begin{document}
	\doublespace
	\title{Detecci\'on de ansiedad por medio de c\'omputo vestible}
	\author{Dari\'en Alberto Miranda Boj\'orquez}
	\maketitle
        \maketitle
        \newpage
        \tableofcontents
        \newpage


                \chapter{Introducci\'on}
                        La ansiedad es una emoci\'on caracterizada por sensaciones de tensi\'on, pensamientos de preocupaci\'on y cambios f\'isicos como incremento en la presi\'on arterial.\citep{psychologyapa}.Una forma com\'un en donde la ansiedad se presenta es en el estr\'es. La relaci\'on entre el estr\'es y la ansiedad es la ansiedad es la se\~nal psicofisiol\'ogica de que la respuesta al estr\'es ha sido iniciada \citep{PMID2235645}.Es com\'un que la poblaci\'on en general tenga episodios de ansiedad debido a los tipos de trabajo de nuestra sociedad moderna. Durante esos lapsos de tiempo, la persona suele experimentar un nivel de ansiedad el cual es una reacci\'on normal para lograr objetivos. Sin embargo, cuando la persona experimenta un nivel de ansiedad el cual es tan alto que no le permite manejar su vida normal, se dice que la persona tiene un desorden de ansiedad\citep{repetto2013}.


                        Por otro lado, el c\'omputo vestible nos permite llevar computadoras con nosotros de la misma manera que llevamos la ropa puesta. Al ``vestir" un dispositivo,
                        el usuario tiene acceso a una computadora que es capaz de monitorearlo a \'el y a su entorno por medio de sensores. Los sensores pueden medir entre
                        otras cosas: movimientos del cuerpo del usuario, la posici\'on del usuario, intensidad de luz, ruido, im\'agenes de su ambiente, ritmo card\'iaco, capacidad
                        conductiva de la piel, distancias, entre otros. Debido a su caracter\'istica de ser vestible, se pueden hacer monitoreos constantes y mas precisos que con
                        los sistemas tradicionales y ayudar en las tareas de la vida cotidiana.

                        El uso de c\'omputo vestible para la detecci\'on de la ansiedad abre una posibilidad para ayudar a reducir el riesgo a la salud mental
                        de los cuidadores de pacientes con autismo o demencia. La siguiente secci\'on ejemplifica posibles escenarios de aplicaciones reales

			Aplicaciones

                        Uno de los sectores de poblaci\'on vulnerables, son los cuidadores de pacientes con autismo o demencia. Se encuentra documentado que los cuidadores, al llevar una carga f\'isica, cognitiva y emocional derivada de su labor les genera padecimientos como ansiedad, estr\'es, y hasta la muerte\citep{Chen2013}. Debido a que los cuidadores no necesariamente son personas con una formaci\'on profesional, estos efectos pueden verse aumentados. Por lo general, los cuidadores que son familiares del paciente son a\'un m\'as afectados ya que necesitan administrar el tiempo de trabajo, familia, actividades sociales y la actividad misma del cuidado del paciente.
			%Cuales son las situaciones ( escenarios ) en los que los CUIDADORES presentan ansiedad?
                        %autismo
                        El desorden de autismo (referido como autismo) es una de las variedades del Espectro de Des\'ordenes del Autismo (ASD) y est\'a caracterizado por interacciones sociales da\~nadas, ausencia de habilidades de comunicaci\'on, movimientos estereotipados y mal comportamiento en general\citep{bernier2010autism}. Existen escuelas especiales para ni\~nos con este padecimiento.

                        %demencia
La demencia es un s\'indrome del declive de las habilidades cognitivas. Los s\'intomas comunes son: problemas de memoria, dificultades para realizar tareas familiares, mal juicio, deterioro del lenguaje hablado y cambios de humor\citep{Aziz}. Afecta alrededor de el 4\% de las personas mayores de 65 a\~nos y al 40\% de las personas mayores de 90.


	\chapter{Objetivos}
		
	\section{\textbf{Objetivo general:}}
			Proponer una t\'ecnica para la detecci\'on de ansiedad basada en c\'omputo vestible para detectar periodos de ansiedad

	\section{\textbf{Objetivos espec\'ificos:}}
		\begin{itemize}
			\item Desarrollar un m\'etodo basado en sensado oportunista que permita recolectar datos relacionados con ansiedad en cuidadores.
			\item Proponer una t\'ecnica para cuantificar niveles de ansiedad.
			\item Comparar la t\'ecncia con los m\'etodos de auto reportado.
		\end{itemize}
	\chapter{Fundamentos te\'oricos}
	\section{Qu\'e es la ansiedad?}
	
		La ansiedad es un \textit{estado mental} caracterizado por pensamientos de preocupaci\'on y de aprehensi\'on acerca de una "amenaza" presente o futura. Si bien, esta amenaza puede ser reconocida por la persona, en muchas ocasiones es dif\'icil de reconocer \citep{Bystritsky2014489}. En su lado fisiol\'ogico, la ansiedad genera un aumento en la frecuencia cardiaca, sudoraci\'on, respiraci\'on y dilataci\'on de las pupilas entre otras manifestaciones. 

		\subsection{Ansiedad y estr\'es}	

		Si bien la ansiedad y el estr\'es son dos estados mentales muy parecidos (ambos comparten las mismas manifestaciones fisiol\'ogicas ) y a veces t\'erminos usados indistintamente, existen ciertas diferencias. El estr\'es es generado cuando la demanda de una tarea es mas grande que la percecpci\'on que tiene la persona de sus propias habilidades (cita de internet). Por lo tanto, el estr\'es tiene una fuente siempre \textit{identificable} a diferencia de la ansiedad. El estr\'es se genera en base a eventos pasados presentes y la ansiedad en eventos presentes futuros. Estos estados pueden afectar el uno al otro, es decir, la ansiedad puede generar estr\'es tanto como el estr\'es puede generar ansiedad.

			Existen dos tipos de ansiedad reconocidos por la \textit{American Psychological Association}. La \textbf{State Anxiety} y la \textbf{Trait anxiety}.La State Anxiety, es aquella que se genera debido a un evento presente en espec\'ifico, mientras que la Trait Anxiety es aquella que la persona siente dia a dia. Esta ansiedad debe de ser tan com\'un en la persona que forma parte de su \textit{personalidad}.

		La ansiedad es generada por la forma en que el cerebro procesa el \textit{stimuli}, al hacer prejuicios sobre la informaci\'on recibida. Por lo tanto, un mismo evento puede ser percibido con diferente nivel de ansiedad por dos personas diferentes. Por lo general, la ansiedad es un estado del que las personas no tienen mucho control. Esto es debido a que es el cerebro delega las manifestaciones al sistema aut\'onomo central y genera las manifestaciones fisiol\'ogicas.

	\section{M\'etodos para la deteccion de ansiedad tradicionales}
	En psicolog\'ia existen diversos instrumentos que permiten detectar la ansiedad. Algunos de los cuestionarios existentes son:
	\begin{itemize}
		
		\item Hamilton Anxiety Rating Scale

		\item Zung Self-Rating Anxiety Scale (SAS)
				
		\item The State-Trait Anxiety Inventory (STAI)
		\item Subjective Self-raiting Anxiety Scale (SUDS)
	\end{itemize}

	\section{Estrategias para el afrontamiento (coping strategies) de la ansiedad en cuidadores}
	Existen diversas estrategias  para el manejo de la ansiedad. Una forma de catalogarlo es en tres grupos\citep{Cooper200615}:
	
		\begin{itemize}
			\item Enfocadas en los sentimientos
			\item Basadas en la resoluci\'on de problemas
			\item Estrategias disfuncionales
		\end{itemize}
	
	Existen  relaciones entre las estrategias usadas por los cuidadores y el estado mental del cuidador. Cuando las t\'ecnias enfocadas en la resoluci\'on de problemas decrementan, y el uso de las t\'ecnicas disfuncionales aumenta, la carga del cuidador aumenta. Monitorear las estrategias de afrontamiento pueden ayudar a predecir la ansiedad, carga y niveles de carga en el cuidador\citep{Tutar2013P484}. 
	\section{M\'etodos para la deteccion de ansiedad por medio de computadoras}
		El aprendizaje de m\'aquina (Machine Learning) es una t\'ecnica de ciencias de la computaci\'on que busca aprender informaci\'on a partir de datos.
		Existen diversas formas en las que se puede hacer aprendizaje de m\'aquina. A continuaci\'on se muestran los m\'etodos mas comunes para la detecci\'on de emociones:
		\section{M\'aquinas de soporte vectorial}
			Las m\'aquinas de soporte vectorial \textit{(Vectorial Support Machines)} son sistemas que nos permiten clasificar caracter\'isticas de datos en diferentes clases.
		
	\newpage
		\chapter{Metodolog\'ia}
		Para lograr la detecci\'on de ansiedad, se desarrollaron dos fases distintas. En la primera, utilizamos cuidadores de perros \textit{(Canis lupus familiaris)} para colectar datos en situaciones naturalistas para desarrollar una primera aproximaci\'on de nuestro modelo. En la segunda, adaptamos y aplicamos el modelo desarrollado para detectar ansiedad en cuidadores de personas con alzheimer.
		\section{Fase 1:} 
			TODO: Explicar la relaci\'on que existe entre la ansiedad previa en el cuidador y el animal.
			
	\newpage
		\chapter{Mis notas}
			Esta secci\'on no es parte de la tesis. Aqu\'i agrego las citas que considero interesantes para mi tesis y que a\'un no encuentro el lugar adecuado donde ponerlas.

	
	
		En un estudio, 60\% de los cuidadores desarrollaron un desorden depresivo y/o de ansiedad en los primeros 24 meses: 37\% de depresi\'on, 55\% desorden de ansiedad y 32\% ambos. \citep{Joling2014}

		Casi un cuarto de los cuidadores de personas con demencia tienen un nivel de ansiedad cl\'inico significante \citep{Cooper200615}

		Mientras que la carga de los cuidadores informales aumenta, se vuelve mas probable que sufran de ansiedad y depresion \citep{Denno20131731}

		La carga en los cuidadores (f\'isica o psicol\'ogica) podr\'ia aumentar los niveles de ansiedad. Entre mas demandante es un servicio, mayor podr\'ia ser la ansiedad percibida. Los comportamientos bizarros o impredecibles de la persona afectada por demencia aumentan la carga emocional \citep{Rosa201054}

		Induciendo ansiedad: Una manera que se puede inducir ansiedad es por medio de situciones generadas. Podriamos poner a una persona en una situacion virtual de ansiedad mientras que sus signos
		vitales son monitorizados. Al dar una retroalimentacion biologica, hacemos consciente al usuario de tal manera que la ansiedad sea inducida.

		Como medirlo: Por medio de la disminucion de la ansiedad y la revision del bienestar percibido de la persona
		
		Los aspectos cognitivos se encuentran conectados a los sentimientos de la persona \citep{Wilt2011987}. Por lo que su estado emocional puede afectar en la percepci\'on de la dificultad de la tarea a realizar.

		H1: \textit{Los sujetos con un nivel de ansiedad mayor encontrar\'an mas dif\'icil de hacer una tarea altamente demandante que aquellos con un nivel de ansiedad bajo}.
		

		H2: \textit{Los sujetos con un nivel de ansiedad mayor tendr\'an valores de HR, SBR ( twitching ) y  de ST aumentar\'an durante una tarea altamente demandante}.
		\newpage
		
		Las frecuencias beta en los estudios de EEG son las caracteristicas mas relacionadas con el estres \citep{Sharma20121287}
		Algunas de las actividades que realizan los cuidadores de personas con demencia son las siguientes \citep{tagkey2008110}: 
		\begin{itemize}
			\item Comprar alimentos, preparar comidas y proveer transporte.
			\item Ayudar a la persona a tomar medicinas correctamente y seguir las recomendaciones de los tratamientos para su demencia y otras condiciones m\'edicas.
			\item Administrar las financias y asuntos legales.
			\item Supervisar a la persona para evitar actividades peligrosas, como deambular y perderse.	
			\item Ba\~nar, vestir, alimentar y ayudar a la persona a usar el ba\~no o proveer cuidados para la incontienencia.
			\item Hacer citas para cuidados m\'edicos.
			\item Lidiar con s\'intomas de comportamiento.

		\end{itemize}
		\textbf{Cuestionarios para medir la severidad de la demencia:}
		\begin{itemize}

			\item Minimental State Examination MMSE

		\end{itemize}
		\textbf{Cuestionarios para medir la carga en el cuidador:}
		\begin{itemize}

			\item Escala de Sobrecarga del Cuidador de Zarit

		\end{itemize}
		\textbf{Drogas para calmar la ansiedad:}
		\begin{itemize}
			\item Xanax (alprazolam)
			\item Valium (diazepam)
			\item Klonopin (clonazepam)
			\item Ativan (lorazepam)
		\end{itemize}

	El uso de dispositivos vestibles representa una oportunidad para detectar la ansiedad \citep{Miranda:2014:ADU:2676690.2676694}
	
	Las situaciones menos controlabes pueden ser manejadas al distanciarse ocasionalmente del estresador ( el paciente )  y cambiando la respuesta emocional a trav\'es de estrategias basadas en la emoci\'on
	\textbf{Coping techniques}

	\citep{li20121}
	\begin{itemize}
		\item Evitar la situaci\'on.
	\end{itemize}

	Esta nota es para mostrar el modelo de generacion de estres por medio del GSR \citep{Bakker2011}
	Esta nota es para mostrar la cita de Rani sobre deteccion de ansiedad por medio de logica difusa\citep{Rani2007323}
        {\normalsize
		\bibliographystyle{cicese}
                \bibliography{thesis}
        }

	\end{document}
