Due to the increase of the older adults population with dementia, the quantity of caregivers has increased. Anxiety is one of the top problems that affect to caregivers. Current studies are based in traditional techinques like polls or questionaries and only show results in long term. Wearable computing (computers small enough to be used as clothes) have the capacity to sense the user at all time. This opens the opportunity to detect anxiety spans at the moment. Despite the fact that anxiety problems are worse in long term, caregivers may be benefited by the state anxiety detection. In this work, an experiment design to gather caregivers of dementia anxiety data  in naturalistic situations and a technique to detect anxiety is reported. In this study, 10 subjects participated in 3 sessions of cognitive therapies with a person acting as if had dementia. A physiological database was gathered. This database contains data as galvanic skin response, hearth rate, accelerometer and electroencephalogram among others. A preliminar study is conducted to show how wearable computing is fit to detect anxiety spans in caregivers of people with dementia.
