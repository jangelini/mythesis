Anxiety is one greatest problems that affect to the caregivers of people with dementia. Current studies are based in traditional techinques like polls or questionaries and only show results in long term. Wearable computing (computers small enough to be used as clothes) have the capacity to sense the user at all time. This opens the opportunity to detect anxiety spans at the moment. Despite the fact that anxiety problems are worse in long term, caregivers may be benefited by the state anxiety detection. In this work, an experiment design to gather caregivers of dementia anxiety data in simulated situations is reported. In this study, 10 persons participated in 3 sessions of cognitive therapies with a person acting as if had dementia. Participants were told that the actor was an real person with dementia. A physiological database of 974.64 minutes of data was gathered. This database contains data as galvanic skin response, hearth rate, accelerometer and electroencephalogram among others. A preliminar study is conducted to show how wearable computing is fit to detect anxiety spans in caregivers of people with dementia.
