Anxiety is one greatest problems that affect the caregivers of people with dementia. Current studies are based on traditional techinques like polls or questionaries and only show results in long term. Wearable computing (computers small enough to be used as clothes) have the capacity to sense the user at all time. This opens the opportunity to detect anxiety spans at the moment when it is experienced by the individual. Despite the fact that anxiety problems are worse in the long term, caregivers may benefit from the detection of state anxiety. In this work, an experiment design to gather caregivers of dementia anxiety data in simulated situations is reported. In this study, 10 persons participated in 3 sessions of cognitive therapies with a person acting as it had dementia. Participants were told that the actor was an diagnosed with dementia. A physiological database of 974.64 minutes of data was gathered. This database contains data of galvanic skin response, hearth rate, accelerometer and electroencephalogram. A preliminar study is conducted to show the feasibility of using wearable computing to detect axiety spans in caregivers of people with dementia.
